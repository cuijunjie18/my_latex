\documentclass[12pt, a4paper, oneside]{ctexart}
\usepackage{amsmath, amsthm, amssymb, graphicx}
\usepackage[bookmarks=true, colorlinks, citecolor=blue, linkcolor=black]{hyperref}

% 导言区

\title{我的第一个\LaTeX 文档}
\author{崔俊杰}
\date{\today}
\newtheorem{theorem}{勾股定理}[section]


% 正文区

\begin{document}

\maketitle

\tableofcontents

\section{学习一}

\subsection{二级标题}

这里是正文. 

\subsection{二级标题}

这里是正文. 

\subsection{原神美图展示}
见下图

\begin{figure}[htbp]
    \centering
    \includegraphics[width=8cm]{影.jpg}
    \caption{雷电将军}
\end{figure}

\section{学习二}

\subsection{表格学习}

\begin{table}[htbp]
    \centering
    \caption{测试}
    \begin{tabular}{ccc}
        1 & 2 & 3 \\
        4 & 5 & 6 \\
        7 & 8 & 9 \\
    \end{tabular}
\end{table}

\begin{tabular}{|l|l|}
    \hline
    迪卢克 & 火 \\
    \hline
    夜兰 & 水 \\ 
    \hline
    艾尔海森 & 草 \\ 
    \hline
\end{tabular}

\subsection{列表学习}
\begin{enumerate}
    \item 一阵微风塑起记忆
    \item 吹过森林
    \item 萤火的痕迹
\end{enumerate}

\subsection{列表自定义样式}
\begin{enumerate}
    \item[(1)] 一阵微风塑起记忆
    \item[(2)] 吹过森林
    \item[(3)] 萤火的痕迹
\end{enumerate}

\section{学习三}

\subsection{定理展示}
\begin{theorem}[勾股定理]
\end{theorem}

\subsection{字体大小}
normal size words

{\tiny tiny words}

{\scriptsize scriptsize words}

{\footnotesize footnotesize words}

{\small small words}

{\large large words}

{\Large Large words} 

{\LARGE LARGE words}

{\huge huge words}

\end{document}